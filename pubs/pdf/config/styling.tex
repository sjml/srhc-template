\usepackage[hyperfootnotes=false,unicode,linktoc=all]{hyperref}
\hypersetup{
  pdftitle={Nulla Laborum Exercitation Nostrud},
  pdfauthor={Lorem Ipsum},
  pdflang={en-US},
  pdfcreator={LaTeX + pandoc + custom templates},
  colorlinks=true,
  linkcolor=Black,
  filecolor=Black,
  citecolor=Blue,
  urlcolor=blue!80!black
}

% \usepackage{fontspec}
\usepackage[bidi=default]{babel}
\babelfont{rm}[
    Path           = ../../resources/fonts/Alegreya/static/,
    Extension      = .ttf,
    UprightFont    = *-Regular,
    ItalicFont     = *-Italic,
    BoldFont       = *-Bold,
    BoldItalicFont = *-BoldItalic
]{Alegreya}

\babelfont[hebrew]{rm}[
    Path           = ../../resources/fonts/DavidLibre/,
    Extension      = .ttf,
    UprightFont    = *-Regular,
    BoldFont       = *-Bold
]{DavidLibre}

\babelfont[latin]{rm}[
    Path           = ../../resources/fonts/Alegreya/static/,
    Extension      = .ttf,
    UprightFont    = *-Italic,
    ItalicFont     = *-Regular,
    BoldFont       = *-BoldItalic,
    BoldItalicFont = *-Bold
]{Alegreya}

\newfontfamily{\headerfont}{AlegreyaSC}[
    Path           = ../../resources/fonts/AlegreyaSC/,
    Extension      = .ttf,
    UprightFont    = *-Regular,
    ItalicFont     = *-Italic,
    BoldFont       = *-Medium,
    BoldItalicFont = *-ExtraBoldItalic
]

\setmonofont[
    Path           = ../../resources/fonts/CascadiaMono/static/,
    Extension      = .ttf,
    UprightFont    = *-SemiLight,
    ItalicFont     = *-SemiLightItalic,
    BoldFont       = *-Bold,
    BoldItalicFont = *-BoldItalic
]{CascadiaMono}


% no numbering on divisions
\setsecnumdepth{none}
% push the part titles to the right
\renewcommand{\parttitlefont}{\headerfont\HUGE\bfseries\raggedleft}
% set chapters and headers to use the font defined above
\setsecheadstyle{\Large\bfseries\raggedright}
\renewcommand*{\chaptitlefont}{\headerfont\Huge\bfseries\centering}

% adjusting the height of the chapter name offset
\setlength{\beforechapskip}{40pt}
% \indentafterchapter % toggling indentation of first paragraph of chapter

% parts shouldn't show page numbers
\aliaspagestyle{part}{empty}
% chapters usually do, but not in the ToC; this gets
%   turned off and on during the main.tex document assembly
% \aliaspagestyle{chapter}{empty}
% \aliaspagestyle{chapter}{plain}

% in general, put the page numbers in the corner and nothing else
\pagestyle{simple}

% don't show page numbers for part headers in the ToC
\cftpagenumbersoff{part}

% indent chapter headings in ToC
\cftsetindents{chapter}{1.5em}{0.0em}

% put dots between chapter names and their page numbers in the ToC
\renewcommand{\cftchapterdotsep}{5}

% chapter names and numbers should just be plain (bold otherwise)
\renewcommand{\cftchapterfont}{}
\renewcommand{\cftchapterpagefont}{}

\usepackage{framed}
\usepackage{mdframed}

% blockquotes should be ragged-right and have a line to the left
\setlength{\FrameSep}{0em}
\setlength{\fboxrule}{2.5pt}

% custom leftbar with minimal space between bar and text
\newenvironment{customleftbar}[1][\parindent]
  {\def\FrameCommand{\hspace{#1}\textcolor[rgb]{0.65, 0.65, 0.65}{\vrule width \fboxrule} \hspace{-1.0em}}%
   \MakeFramed {\advance\hsize-\width \FrameRestore}\vspace{0pt}}%
  {\vspace{0pt}\endMakeFramed}

% redefine the quote environment to include custom leftbar and raggedright
\let\oldquote\quote
\let\endoldquote\endquote
\renewenvironment{quote}
  {\begin{customleftbar}\oldquote\raggedright}
  {\endoldquote\end{customleftbar}}

% put a nice gray box behind our code
\newenvironment{Shaded}{%
  \begin{mdframed}[%
    backgroundcolor=shadecolor, innertopmargin=15pt, innerbottommargin=15pt,
    innerleftmargin=20pt, innerrightmargin=20pt, linewidth=0pt, roundcorner=0pt]}%
  {\end{mdframed}}

% put the gray box around inline code, too
\usepackage{tcolorbox}
\definecolor{codecolor}{RGB}{233,236,239} % $gray-200
\newtcbox{\codebox}[1][]{on line,colback=codecolor,arc=1pt,boxsep=0pt,left=3pt,right=3pt,top=3pt,bottom=3pt,boxrule=0pt}

% yikes, it seems impossible to get a good looking gray box that can
% break across lines. what I've left in place here (using \codebox) looks
% the best. the sethlcolor (using soul) should handle breaks, but pandoc
% always makes every space into an explicit space ("this text here" becomes
% "this\ text\ here") and that breaks soul because it can't calculate the
% width anymore.
% so, going with the better looking one for now, but might need to investigate
% further if this becomes a problem.

% this is what I was looking to do at one point
% \usepackage{soul}
% \renewcommand{\texttt}[1]{{\ttfamily\hl{#1}}}
% \renewcommand{\texttt}[1]{\sethlcolor{codecolor}{\ttfamily\hl{\mbox{#1}}}}

% be able to test if we're inside a footnote
\newif\iffootnoteactive
\footnoteactivefalse
\let\oldfootnote\footnote
\renewcommand{\footnote}[1]{%
  \global\footnoteactivetrue%
  \oldfootnote{#1}%
  \global\footnoteactivefalse%
}
% if we're inside a footnote, be scriptsize; otherwise be small
\renewcommand{\texttt}[1]{{\ttfamily{\codebox{\iffootnoteactive{\scriptsize{#1}}\else{\small{#1}}\fi}}}}

